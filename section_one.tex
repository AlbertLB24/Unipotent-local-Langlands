\subsection{Local Fields and Locally Profinite Groups}
We begin by recalling some basic objects from algebraic number theory. Given a field $F$, a \textbf{discrete valuation} on $F$ is a surjective function $\nu: F\to\ZZ\cup\{\infty\}$ satisfying the three conditions

\begin{enumerate}
    \item $\nu(xy)=\nu(x)+\nu(y)$ for any $x,y\in F$ 
    \item $\nu(x+y)\geq\min\{\nu(x),\nu(y)\}$ for any $x,y\in F$.
    \item $\nu(x)=\infty$ if and only if $x=0$.
\end{enumerate}

Any discrete valuation $\nu$ induces an absolute value given by the formula 
$$|x|=c^{\nu(x)}$$ 
for any $c\in(0,1)$, and therefore it also induces a topology. We remark that this topology is independent of the choice of $c$. In addition, the absolute value satisfies $|x+y|\leq\max\{|x|,|y|\}$ for any $x,y\in K$. Absolute values with this property are denoted as \textit{non-Archimedean}.

A field $F$ with a an absolute value $|\cdot|$ induced by a discrete valuation $\nu$ is the fraction field of the valuation ring
$$R:=\{x\in F:v(x)\geq 0\}=\{x\in K: |x|\leq1\},$$ 
which contains a unique maximal ideal
$$\pp:=\{x\in F:v(x)> 0\}=\{x\in K: |x|<1\},$$
denoted as the valuation ideal. This ideal is principal, and it is generated by any $\varpi\in K$ with $\nu(\varpi)=1$. Such an element is called a uniformizer of $F$. Finally, the residue field $\kappa$ of $F$ is the quotient $R/\pp$. This motivates the following definition.

\begin{defn}
    A field $F$ is a (non-Archimedean) \textit{local field} if it is complete with respect to a topology induced by a discrete valuation and with finite residue field.
\end{defn}

\begin{rem}
    When the residue field is finite, it is conventional to write 
    $$|x|=q^{-\nu(x)},$$ 
    where $q=|\kappa|$. From here onwards, we will follow this convention.
\end{rem}

As discussed above, the valuation ring $R$ of a local field $F$ is a local ring with unique principal ideal $\pp$. Furthermore, the ideals 
$$\pp^n=\{x\in F:\nu(x)\geq n\}=\{x\in F: |x|\leq q^{-n}\}=\varpi^n R,\quad n\in\NN$$
are a complete set of ideals of $R$ and also a fundamental system of neighbourhoods of the identity. Under the topology induced by the discrete valuation, the field $F$ (and therefore also $R$) is totally disconnected, and furthermore we also have a topological isomorphism
$$R\longrightarrow\varprojlim_{n\geq 1} R/\pp^n,$$
where the maps implicit in the right hand side are the obvious ones.
In particular, since the residue field is finite, all rings $R/\pp^n$ are finite and induced with the discrete topology. Hence the inverse limit, being a closed subset of the product of compact sets, is a compact set. This shows that $R$, and therefore also any $\pp^n$ for any $n\in\ZZ$ is a compact open subring of $F$. We have therefore shown that $F$ has the important property that any open subset of $F$ contains an open compact subgroup (namely $\pp^n$ for a sufficiently large $n$). 

We also remark that $F$ satisfies the rather special property of being the union of its open compact subgroups, even though $F$ itself is clearly not. This fact has relevant consequences as we may discuss later.

We are now ready to give the main definition of this section.

\begin{defn}
    A topological group $G$ (which we always assume to be Hausdorff) is a \textit{locally profinite group} if every open neighbourhood of the identity contains a compact open subgroup. 
\end{defn}

In this document we will be interested in studying the representation theory of many important groups and rings related to the local field $F$. The notion of a locally profinite group is an abstract one, but it has the great advantage of accomodating many important groups and rings associated to non-Archimedean local fields and their representation theory.

\begin{examples}

    \begin{enumerate}
        \item In the preceding discussion, we have shown that $F$ is a locally profinite group, where $\pp^n$ for $n\geq1$ is a fundamental system of open compact subgrups.
        \item The multiplicative group $F^{\times}$ is also a locally profinite group, where the congruence unit groups $U_F^n=1+\pp^n$ for $n\geq1$ is a fundamental system of open compact subgroups. We remark that unlike $F$, the group $F^{\times}$ is not the union of its open compact subgroups.
        \item Given $n\geq1$ an integer, the additive group $F^n=F\times\dots\times F$ is also a locally profinite group endowed with the product topology. More generally, the product of locally profinite groups is locally profinite.
    \end{enumerate}
\end{examples}

We give some further insight into the terminology used. If $G$ is a locally profinite group, any open subgroup $K$ of $G$ is also a locally profinite group under the subspace topology. Also, if $H$ is a closed normal subgroup of $G$, then $G/H$ is also locally profinite. Recall that any profinite group is a locally profinite group and it is compact. Using a topological argument, one can also show that the converse also holds. That is, if $K$ is a compact locally profinite group, then
$$K\longrightarrow\varprojlim K/N$$
is a topological isomorphism where $N$ ranges over the normal open subgroups, and the implicit maps are the obvious ones.

\subsection{Continuous Characters of Local Fields}
\subsection{Smooth representations of locally profinite groups}

Let $G$ be a locally profinite group. A representation of $G$ is a pair $(\pi, V)$ where $V$ is a complex vector space (not necessarily finite-dimensional) and $\pi: G \to \GL(V)$ is a homomorphism.
\begin{defn}
	A representation $V$ of $G$ is smooth if for $v\in V$ there exists a compact-open subgroup $K\subseteq G$ such that $kv = v$ for all $k\in K$ (i.e. $v\in V^K$). We say $V$ is admissible if $V^K$ is finite-dimensional for all compact-open $K$.
\end{defn}

Let us define induced representations:
\begin{defn}\label{induction}
	Let $G$ be a locally profinite group, $H\subseteq G$ a closed subgroup, and $W$ a smooth representation of $H$. We define the induced representation as the space of functions $f: G\to W$ satisfying
	\begin{enumerate}
		\item $f(hg) = h\cdot f(g)$ for all $h\in H, g\in G$
		\item there is a compact open subgroup $K\subseteq G$ (depending on $f$) such that $f(gk) = f(g)$ for all $g\in G, k\in K$.
	\end{enumerate}
	We let $G$ act on this space via $(g\cdot f)(x) = f(xg)$. This is a smooth representation of $G$, denoted by $\Ind_H^G W$.
\end{defn}
This construction satisfies the following important universal property:
\begin{thm}[Frobenius reciprocity]
	Let $V$ be a smooth representation of $G$, and $W$ a smooth representation of $H$. Then there's a natural bijection
	\begin{align*}
		\Hom_G(V, \Ind_H^G W)&\cong \Hom_H(V, W)\\
		\varphi &\mapsto \alpha_W \circ \varphi
	\end{align*}
	where $\alpha_W: \Ind_H^G W \to W$ is the canonical map $\alpha_W(f) = f(1)$.
\end{thm}

There is also a different variant of induction, called compact induction:
\begin{defn}
	Let $G$ be a locally profinite group, $H$ a closed subgroup, and $W$ a smooth representation of $H$. Consider the space of functions $f: G\to W$ that satisfy (1) and (2) in Definition \ref{induction}, and are also compactly supported mod $H$, i.e. the support $\mathrm{supp} f\subseteq H\backslash G$ is compact. The group $G$ also acts on this space by right translations, so we get a representation of $G$, denoted by $c-\Ind_H^G W$.
\end{defn}
This construction is mainly of interest in the case when $H$ is open in $G$. In this case, it satisfies a version of Frobenius reciprocity:
\begin{thm}
	Let $H\subseteq G$ be open, $W$ a smooth representation of $H$ and $V$ a smooth representation of $G$. We have a natural bijection
	\begin{align*}
		\Hom_G(c-\Ind_H^G W, V)&\cong \Hom_H(W, V)\\
		\varphi &\mapsto \varphi \circ \alpha^c_W 
	\end{align*}
	Where $\alpha^c_W: W\to c-\Ind_H^G W$ is the map $w\mapsto f_w$ where $f_w$ is supported in $H$ and satisfies $f_w(h) = hw$.
\end{thm}