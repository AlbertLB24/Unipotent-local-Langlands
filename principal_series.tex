\section{Principal series representations of $\GL_2$}

Let $F$ be a nonarchimedean local field, $G=\GL_2(F)$, $B=\{\begin{psmallmatrix} *&*\\0&*\end{psmallmatrix}\}$ the Borel subgroup of upper triangular matrices, so that $B=NT$ for $T=\{\begin{psmallmatrix}
    *&0\\0&*
\end{psmallmatrix}\} \cong F^\times \times F^\times$ and $N=\{\begin{psmallmatrix}
    1&*\\0&1
\end{psmallmatrix}\}\cong F$ with $N \lhd B$. Between $N$ and $B$ we also have the mirabolic subgroup $M=\{\begin{psmallmatrix}
    *&*\\0&1
\end{psmallmatrix}\}$ with $M/N \cong F^\times$.

Ultimately we want to understand irreducible representations of $G$ (for example, modular forms give rise to automorphic forms, the space of which is a smooth representation of $\GL_2(\adele)$, and this decomposes under Flath's theorem to give smooth representations of $G$). Initially this is too difficult so we restrict to simpler subgroups like $B$, and then further to $N$ and $T$, although it is more natural to view $T$ as a quotient of $B$. Then we get representations of $B$ inflating from characters of $T$, and inducting to $G$ is called parabolic induction, giving so-called principal series representations. We want to understand how these decompose into irreducibles, and from there we can classify all (irreducible) principal series representations using Frobenius reciprocity.

To understand the decomposition of parabolically induced representations into irreducibles as $G$-representations, we want to see how they decompose into irreducibles over a less unwieldy subgroup of $G$, such as $B$. It turns out that these do not decompose any further over $M$ than over $B$. On the other hand, the representation theory of $M$ is very easy to classify - this is what makes the mirabolic subgroup so 'miraculous'. To get representations of $M$ we can induct from characters of $N$, or inflate from $M/N\cong F^\times$. There are many characters of $N\cong F$, in fact these are in bijection with $F$ under $\psi(x) \mapsto \psi(ax)$ for $a \in F$ and any nontrivial character $\psi$. The key property of $M$ is that conjugation by $M$ acts transitively on these characters $\psi$, which greatly simplifies the representation theory of $M$ coming from induction from $N$. $M$ is also small enough that this induction, together with the characters of $F^\times$, give all irreducible representations of $M$.

\subsection{Representations of $N$}

For an abelian group, all the irreducible representations are characters (Schur's lemma?), and when the group is finite, any representation decomposes as a direct sum of characters. $N \cong F$ is not finite, so we lose this decomposition, but it is still true that any vector is nonzero in some quotient on which $N$ acts via a character. To formalise this, we define

\begin{notn}
    Let $V$ be a smooth representation of $N$ and $\theta$ a character of $N$. Let $V(\theta) \leq V$ be the subspace spanned by $n\cdot v - \theta(n)v$ for $n\in N$, and set $V_\theta = V/V(\theta)$ so that $N$ acts on $V_\theta$ by $\theta$. When $\theta$ is trivial we write $V(N)$ and $V_N$ respectively. 
\end{notn}

The following is a useful equivalent definition of $V(\theta)$:

\begin{lemma}\label{criteria N}
    The vector $v \in V$ lies in $V(\theta)$ if and only if 
    $$\int_{N_0} \theta(n)^{-1} \cdot n \cdot v dn = 0$$
    for some compact open subgroup $N_0$ of $N$ (we restrict to compact opens for the integral to be well defined).
\end{lemma}
\begin{proof}
    Bushnell-Henniart Lemma 8.1.
\end{proof}

\begin{cor}
    The functo $V \mapsto V_\theta$ is exact from representations of $N$ to complex vector spaces.
\end{cor}
\begin{proof}
    Taking quotients in this way is right exact. So we need to show that if $f: V \hookrightarrow V'$ then $V_\theta \hookrightarrow V'_\theta$. If $v \in V$ with $f(v) \in V'(\theta)$ then 
    $$\int_{N_0} \theta(n)^{-1}n \cdot f(v) dn = 0$$
    for some $N_0$. But this integral is a finite sum and $f$ is compatible with the action of $N$ so that we can pull $f$ out of the integral. Injectivity of $f$ implies
    $$\int_{N_0} \theta(n)^{-1}n \cdot v dn = 0$$
    from which we deduce that $v \in V(\theta)$ by the above lemma.
\end{proof}

\begin{prop}
    For any $v \neq 0$ in $V$, there exists a character $\theta$ of $N$ such that $v \not\in V(\theta)$.
\end{prop}
\begin{proof}
    Bushnell-Henniart Proposition 8.1.
\end{proof}

\begin{cor}
    If $V$ is a smooth representation of $N$ with $V_\theta=0$ for all $\theta$ then $V=0$.
\end{cor}


\subsection{Representations of $M$}

Now we consider $V$ an (irreducible) representation of $M$. Note that $V(N)$ is still a representation of $M$ because $N$ is normal in $M$ ($mn\cdot v - m\cdot v = n'm\cdot v - m\cdot v$ for some $n' \in N$), and so $V_N$ is also a representation of $M$ (but $V_\theta$ is not). Since $V$ is irreducible, either $V(N)=0$, so that $N$ acts trivially on $V$ and so we just get a character of $F^\times$, or $V(N)=V$. In the latter case $V_N=0$ so we must have $V_\theta \neq 0$ for all nontrivial characters of $N$ (since the $V_\theta$ are conjugate under $M$), so that the $M$-representation $V$ must have infinite dimension. In fact there is only one such $V$, and we can prove more specifically:

\begin{thm}\label{inf dim}
    Let $(\pi,V)$ be an irreducible smooth representation of $M$. Either 
    \begin{itemize}
        \item $\dim V=1$ and $\pi$ is the inflation of a character of $M/N \cong F^\times$, or
        \item $\dim V = \infty$ and $\pi \cong c-\mathrm{Ind}_N^M \theta$ for any nontrivial character $\theta$ of $N$.
    \end{itemize}
\end{thm}

This itself follows from the following theorem. To compare $V$ and $c-\mathrm{Ind}_N^M \theta$ it is more natural to compare $V$ and $\mathrm{Ind}_N^M V_\theta$. By Frobenius reciprocity,
$$\Hom_N(V,V_\theta) \cong \Hom_M(V,\mathrm{Ind}_N^M V_\theta).$$
Let $q_*: V \to \Ind_N^M(V_\theta)$ be the image of the quotient map $V \to V_\theta$.

\begin{thm}\label{mirabolic}
    The $M$-homomorphism $q_*: V \to \Ind_N^M V_\theta$ induces an isomorphism $V(N) \cong c-\Ind_N^M V_\theta$. Moreover, this compact induction is an irreducible representation of $M$.
\end{thm}
\begin{proof}
    Bushnell-Henniart Theorem 8.3.
\end{proof}

%\begin{notn}
    %For the remainder of this section, let $\mathcal W$ denote $\Ind_N^M \theta$ and $\mathcal W^c$ denote $c-\Ind_N^M \theta$.
%\end{notn}

%Before we get to the proof of this theorem we need a result concerning coinvariants of various spaces. To motivate this, 


\subsection{Irreducible principal series representations}

Let $V$ be a smooth representation of $G$. By restriction this gives a representation of $B$, and so does the space of $N$-coinvariants $V_N=V/V(N)$, again because $N$ is normal in $B$. Then $V_N$ inherits a representation $\pi_N$ of $T=B/N$, and we call this the Jacquet module of $V$ at $N$. As shown before, the Jacquet functor $V \mapsto V_N$ is exact.

Parallel to the classical finite field setting, we want to study when $V$ arises from parabolic induction. We have the analogous result:

\begin{prop}
    The following are equivalent:
    \begin{itemize}
        \item $V_N \neq 0$
        \item $\pi$ is isomorphic to a $G$-subrepresentation of $\Ind_B^G \chi$ for some character $\chi$ of $T$ inflated to $B$.
    \end{itemize}
\end{prop}
\begin{proof}
    (2) implies (1) comes from Frobenius reciprocity:
    $$\Hom_G(\pi,\Ind \chi) = \Hom_B(\pi, \chi) = \Hom_T(\pi_N,\chi)$$ where the second equality is due to any $B$-homomorphism $\pi \to \chi$ factoring through $\pi_N$ (because $\chi$ is trivial on $N$).

    Given (1), one shows by a technical argument that $V_N$ is finitely generated as a representation of $T$. An application of Zorn's lemma allows us to construct a maximal $T$-subspace $U$ of $V_N$ so that $V_N/U$ is an irreducible $T$-representation and is thus a character (Schur's lemma) $\chi$. Frobenius reciprocity implies the result.
\end{proof}
\begin{rem}
    The same proof holds for the finite field case (noting the notion of having a subrepresentation where $N$ acts trivially is the same as having a nonzero quotient where $N$ acts trivially). The proof that (1) implies (2) bypasses the technical details because $V_N$ as a representation of $T$ obviously admits an irreducible quotient as $V_N$ is finite dimensional.
\end{rem}
\begin{rem}
    In the general case we ask for a nonzero quotient of $V$ on which $N$ acts trivially as opposed to having a subrepresentation, because one can show in this latter case that all we get are characters $\pi = \phi \circ \det$ for some character $\phi$ of $F^\times$. In fact any finite dimensional smooth representation is of this form. The difference with the finite field case is that smoothness tells us that if $v\in V$ is fixed by $N$, it is also fixed by an open compact subgroup of $G$. Over a finite field, $N$ is open, but in general it is not and we fix $v$ by too much (all of $\SL_2$).
\end{rem}

We restrict our attention to principal series representations and want to understand how $\Ind_B^G \chi$ decomposes into irreducible $G$-representations. As mentioned earlier, we will first study how they decompose as representations of $B$ or even $M$. 

These induced representations will never be irreducible over $B$ because we always have the canonical $B$-homomorphism $X=\Ind_B^G \chi \to \chi$ given by sending $f \mapsto f(1) \in \CC$. So we have an exact sequence of $B$-representations
$$\xymatrix{0 \ar[r] & V \ar[r] & X \ar[r] & \CC \ar[r] & 0}$$
where $V=\{f \in X \chi : f(1)=0\}$, with $B$ acting on $\CC$ via $\chi$. Now we want to understand how $V$ decomposes. We have another exact sequence of $B$-representations,
$$\xymatrix{0 \ar[r] & V(N) \ar[r] & V \ar[r] & V_N \ar[r] & 0}$$
so we reduce to studying $V(N)$ and $V_N$. We will show that $V(N)$ is irreducible over $B$ (and even over $M$), while $V_N$ will be determined by the Restriction-Induction lemma (which generally treats the exact sequence obtained by applying the Jacquet functor to the first exact sequence, where we may replace $\chi$ by any smooth representation $\sigma$ of $T$).

Firstly we want to understand $V=\{f \in X:f(1)=0\}$ better.

\begin{lemma}\label{reduce to N}
    For $V$ as above, the map 
    \begin{equation*}
        \begin{split}
            V &\to C_c^\infty(N) \\
            f(-) &\mapsto f(w-) 
        \end{split}
    \end{equation*}
    is an $N$-isomorphism, where $w=\begin{psmallmatrix}
        0&1\\1&0
    \end{psmallmatrix}$
\end{lemma}
\begin{proof}
    We have the decomposition $G=B \sqcup BwN$. Since $f(1)=0$ and $f$ is induced from $B$ we must have that $f$ is supported on $BwN$. $G$-smoothness of $f$ implies that $f(1)=0$ is fixed by right translation by some compact open subgroup $K\leq G$. This will contain $\begin{psmallmatrix}
        1&0\\\pi^n \mathcal O &0
    \end{psmallmatrix}$ for some $n$, so that $f$ vanishes on 
    $$\begin{psmallmatrix}
        1&0\\x&1
    \end{psmallmatrix} \in Bw \begin{psmallmatrix}
        1&x^{-1}\\0&1
    \end{psmallmatrix}$$
    for all $x \in \pi^n \mathcal O$. Thus $f(w-)$ is supported on $\begin{psmallmatrix}
        1&y\\0&1
    \end{psmallmatrix} \in N$ with $v(y) > -n$ (so $y \in \pi^{-n}\mathcal O$). $G$-smoothness of $f$ also implies that $f(w-)$ is $N$-smooth, and that the above map is an $N$-homomorphism. The decomposition $G=B \sqcup BwB$ implies that it is in fact an isomorphism.
\end{proof}

\begin{prop}
    For $V$ as above, $V(N)$ is irreducible over $M$ (and hence over $B$).
\end{prop}
\begin{proof}
    By the above lemma we can identify $V \cong C_c^\infty(N)$ with $M$ acting via right translation on $V$. This gives the structure of a $M$-representation on $C_c^\infty(N)$. We can calculate it explicitly (but we won't need it) where
    $$f(bw\begin{psmallmatrix}
        1&x\\0&1
    \end{psmallmatrix}\begin{psmallmatrix}
        a&0\\0&1
    \end{psmallmatrix}) = f(b \begin{psmallmatrix}
        1&0\\0&a
    \end{psmallmatrix}w\begin{psmallmatrix}
        1&a^{-1}x\\0&1
    \end{psmallmatrix})$$
    tells us that the corresponding $M=F^\times N$ action on $C_c^\infty(N)$ is the composite of right translation by $N$ with the action 
    $$a\phi \begin{psmallmatrix}
        1&x\\0&1
    \end{psmallmatrix} = \chi_2(a) \phi \begin{psmallmatrix}
        1&a^{-1}x \\ 0&1
    \end{psmallmatrix}$$

    So now we may consider $V=C_c^\infty(N)$. The benefit is that for this representation, the spaces of coinvariants of characters of $N$ are very simple. In particular, the map $f \mapsto \theta f$ is a linear automorphism of $C_c^\infty(N)$ taking $V(N)$ to $V(\theta)$ since $$n \cdot f - f \mapsto \theta (n \cdot f) - \theta f = \theta(n)^{-1} n \cdot (\theta f) - \theta f \in V(\theta).$$
    Hence all the $V_\theta$ have the same dimension as $V_N=V/V(N)$, which has dimension 1 (we can see this from the characterisation of $V(N)$ as the zeros of some integral, or from the Restriction-Induction lemma to follow).

    But then Theorem \ref{mirabolic} implies that for our $M$-representation $V$, we have $V(N) \cong c-\mathrm{Ind}_N^M V_\theta$ where $V_\theta \cong \theta$ as it is one dimensional. This is irreducible as a $M$-representation by the same Theorem.
\end{proof}

We turn our attention to $V_N$ where we recall $V$ fits in the exact sequence
$$\xymatrix{0 \ar[r] & V \ar[r] & X = \mathrm{Ind}_B^G \chi \ar[rr]^{f \mapsto f(1)} && \chi \ar[r] & 0}$$ of smooth representations of $B$. Since the Jacquet functor is exact, we get the exact sequence

$$\xymatrix{0 \ar[r] & V_N \ar[r] & X_N \ar[r] & \chi \ar[r] & 0}$$ of $T$-representations. We can say in more generality,

\begin{lemma}[Restriction-Induction Lemma]
    Let $(\sigma, U)$ be a smooth representation of $T$ and $(\Sigma, X) = \mathrm{Ind}_B^G \sigma$. Then there is an exact sequence of smooth $T$ representations:
    $$\xymatrix{0 \ar[r] & \sigma^w \otimes \delta_B^{-1} \ar[r] & \Sigma_N \ar[r] & \sigma \ar[r] & 0}$$
\end{lemma}
\begin{proof}
    The proof of Lemma \ref{reduce to N} generalises to show that the vector space $V$ is isomorphic to the space $\mathcal S$ of smooth compactly supported functions $N \to U$ by identifying $f$ with $f(w-)$.

    We can define a map $\mathcal S \to U$ by 
    $$g=f(w-) \mapsto \int_N g(n)=f(wn) dn$$ where this integral is finite since $g$ is compactly supported. By Lemma \ref{criteria N}, this induces an isomorphism $\mathcal S_N \cong U$.

    The $B$-representation structure on $\mathcal S$ coming from $V$ is by right translation, where $b=sm \in TN$ acts by
    $$f(wnsm) = f(wss^{-1}nsm) = f(wswws^{-1}nsm) = \sigma(s^w) f(ws^{-1}nsm)$$
    where $s^{-1}nsm \in N$. Under the isomorphism $\mathcal S_N \cong U$, this induces a $T$ representation structure on $U$ where $s \in T$ acts by 
    $$s \cdot \int_N f(wn) dn = \sigma(s^w) \int_N f(w s^{-1}ns) dn = \sigma(s^w) |\frac{s_1}{s_2}| \int_N f(wn) dn$$ which is $\sigma^w \otimes \delta_B^{-1}$.
\end{proof}

\begin{cor}
    As a representation of $B$ or $M$, $\mathrm{Ind}_B^G \chi$ has composition length 3. Two of the factors have dimension 1, and the other is infinite dimensional.
\end{cor}
\begin{proof}
    This follows from the exact sequences
    $$\xymatrix{0 \ar[r] & V \ar[r] & \mathrm{Ind}_B^G \chi \ar[r] & \chi \ar[r] & 0}$$
    and
    $$\xymatrix{0 \ar[r] & V(N) \ar[r] & V \ar[r] & V_N \ar[r] & 0}$$
    where we saw that $V(N)$ is irreducible (and infinite dimensional by Theorem \ref{inf dim}), and $V_N \cong \chi^w \otimes \delta_B^{-1}$.
\end{proof}

So we understand how $\mathrm{Ind}_B^G \chi$ decomposes into irreducible $B$ representations, and we want to understand its decomposition into $G$ representations. Our goal is to prove the following

\begin{thm}[Irreducibility Criterion]
    Let $\chi = \chi_1 \otimes \chi_2$ be a character of $T$ and let $X = \mathrm{Ind}_B^G \chi$.
    \begin{enumerate}
        \item $X$ is irreducible if and only if $\chi_1\chi_2^{-1}$ is either the trivial character of $F^\times$, or the character $x \mapsto |x|^2$ of $F^\times$.
        \item Suppose $X$ is reducible, then \begin{itemize}
            \item the $G$-composition length of $X$ is 2
            \item one factor has dimension 1, the other is infinite dimensional
            \item $X$ has a 1-dimensional $G$-subspace exactly when $\chi_1\chi_2^{-1}=1$
            \item $X$ has a 1-dimensional $G$-quotient exactly when $\chi_1\chi_2^{-1}(x) = |x|^2$.
        \end{itemize}
    \end{enumerate}
\end{thm}

We make some comments in preparation for the proof. By the above Corollary, if $X$ is reducible then it has a finite dimensional (dimension 1 or 2) $G$-subspace or $G$-quotient. By taking duals we can assume we are in the first case. In the Irreducibility Criterion, we want to show that this implies $\chi_1 = \chi_2$ and that $X$ has a 1-dimensional $G$-subspace.

\begin{defn}
    Let $\pi$ be a smooth representation of $G$ and $\phi$ a character of $F^\times$. The twist of $\pi$ by $\phi$ is the representation $\phi\pi$ of $G$ defined by 
    $$\phi \pi(g) = \phi (\det g)\pi(g).$$
    In this way, for a character $\chi=\chi_1 \otimes \chi_2$ of $T$, we have $\phi\chi = \phi\chi_1 \otimes \phi\chi_2$. Then 
    $$\mathrm{Ind}_B^G(\phi\chi) = \phi \mathrm{Ind}_B^G \chi.$$
\end{defn}

\begin{prop}
    The following are equivalent:
    \begin{enumerate}
        \item $\chi_1=\chi_2$
        \item $X$ has a 1-dimensional $N$-subspace.
    \end{enumerate}
    If this holds then this subspace is unique, and is also a $G$-subspace of $X$ not contained in $V$.
\end{prop}
\begin{proof}
    (1) implies (2): since induction commutes with twisting we may assume $\chi_1=\chi_2=1$, then the nonzero constant function spans a 1-dimensional $G$-subspace (not just $N$-subspace) of $X = \mathrm{Ind}_B^G 1$.

    (2) implies (1): suppose this subspace is spanned by $f$. $N$ acts by right translation as a character. We cannot have $f \in V$ ($f(1)=0$) else we earlier saw that $f$ would then have support in some $BwN_0$ for $N_0 \leq N$ open compact, and this is not closed under multiplication by $N$.

    So $f \not\in V$ ($f(1) \neq 0$) and so its image spans $X/V \cong \CC$ on which $N$ acts trivially (since we inflate $\chi$ to be trivial on $N$). Thus $N$ fixes $f$ under right translation. $f$ is also fixed under right translation by some compact open of $G$, so for sufficiently large $|x|$ we have
    \begin{equation*}
        \begin{split}
            f(w) = f\left(w \begin{psmallmatrix}
                1&x \\0 & 1
            \end{psmallmatrix}\right) &= f\left( \begin{psmallmatrix}
                1&x^{-1} \\0 & 1
            \end{psmallmatrix}\begin{psmallmatrix}
                -x^{-1}&0 \\0 & x
            \end{psmallmatrix}\begin{psmallmatrix}
                1&0 \\x^{-1} & 1
            \end{psmallmatrix}\right) \\
            &= f\left( \begin{psmallmatrix}
                1&x^{-1} \\0 & 1
            \end{psmallmatrix}\begin{psmallmatrix}
                -x^{-1}&0 \\0 & x
            \end{psmallmatrix}\right) \\
            &= \chi_1(-1) \left( \chi_1^{-1}\chi_2(x)\right) f(1)
        \end{split}
    \end{equation*}
    For this to hold for all $|x|$ sufficiently large, it follows that we must have $\chi_1=\chi_2$ (if $\chi_1(y) \neq \chi_2(y)$ then $\chi_1(xy) \neq \chi_2(xy)$ for all sufficiently large $x$, but then $xy$ is also large). The uniqueness of the 1-dimensional subspace comes from the fact that it must span $X/V \cong \CC$. 
\end{proof}

\begin{proof}[Proof of Irreducibility Criterion]
    Assume that $X$ is reducible and we are in the case that $X$ has a finite dimensional $G$-subspace. Then it has a 1-dimensional $N$-subspace $L$, which is also a $G$ subspace by the above Proposition with $G$ acting via $\phi \circ \det$, where $\phi=chi_1=\chi_2$. Since $L \cap V =0$, we see that $Y=X/L \cong V$ as $B$-representations. We need to show $X$ has $G$-length 2. By the previous corollary it has length at most 3. We know that $V$ has $B$-length 2 with a 1-dimensional quotient $V_N$. Thus if $Y$ had $G$-length 2, then the $B$-factors of $V$ are also $G$-factors, so that $G$ must act on $V_N$, necessarily by a character $\phi' \circ \det$ (see 9.2 Exercise 2). But this is impossible because $B \leq G$ acts by $\phi \delta_B^{-1}$ by restriction-induction, and this does not factor through $\det$ on $B$. So we must have that $X$ has $G$-length 2.

    In the other case we have a finite dimensional $G$-quotient. The smooth dual $X^\vee$ is then in the first case, where the Duality Theorem tells us $X^\vee \cong \mathrm{Ind}_B^G \delta_B^{-1} \chi^\vee$. If we write $\delta_B^{-1} \chi^\vee = \psi_1 \otimes \psi_2$ then we must have $\psi_1 = \psi_2$. The result follows from computing $\psi_1(x) = |x|^{-1} \chi_1(x)$ and $\psi_2(x) = |x| \chi_2(x)$.

    The converse direction to (1) follows from the previous Proposition.
\end{proof}


\subsection{Classification of principal series representations}


Now that we've seen how parabolically induced representations decompose into irreducibles, we want to classify the isomorphism classes.

\begin{prop}
    Let $\chi, \xi$ be characters of $T$. The space $\Hom_G(\mathrm{Ind}_B^G \chi, \Ind_B^G \xi)$ is 1-dimensional if $\xi = \chi$ or $\chi^w \delta_B^{-1}$ and 0 otherwise.
\end{prop}
\begin{proof}
    Frobenius reciprocity tells us
    $$\Hom_G(\mathrm{Ind}_B^G \chi, \Ind_B^G \xi) \cong \Hom_T((\Ind \chi)_N, \xi).$$
    From restriction-induction we have
    $$\xymatrix{0 \ar[r] & \chi^w \delta_B^{-1} \ar[r] & (\Ind \chi)_N \ar[r] & \chi \ar[r] & 0.}$$
    In the case $\chi \neq \chi^w \delta_B^{-1}$ the sequence splits and the result follows. If $\chi = \chi^w \delta_B^{-1}$ then $\chi_1\chi_2^{-1} (x) = |x|$ so $\Ind \chi$ is irreducible and the result still follows.
\end{proof}

\begin{rem}
Hence, in the case $\Ind \chi$ is irreducible, we have $\Ind \chi \cong \Ind \chi^w \delta_B^{-1}$.

And in the case $\Ind \chi$ is reducible, it is not semisimple, else the Hom space would be 2-dimensional.
\end{rem}

We can be more explicit in the reducible case. One can check that the conditions in the Irreducibility Criterion of reducibility are equivalent to $\chi$ being of the form $\chi = \phi 1_T$ or $\chi =\phi \delta_B^{-1}$. Untwisting, we may as well assume $\phi=1$.

\begin{defn}
    The Steinberg representation is defined by the exact sequence
    $$\xymatrix{0 \ar[r] & 1_G \ar[r] & \Ind_B^G 1_T \ar[r] & \mathrm{St}_G \ar[r] & 0}$$ which is an infinite dimensional irreducible representation with Jacquet module $(\mathrm{St}_G)_N \cong \delta_B^{-1}$ by restriction-induction. If $\chi =\phi 1_T$ we would instead get a twist of Steinberg, $\phi \mathrm{St}_G$.
\end{defn}

The case $\chi = \delta_B^{-1}$ can be dealt with by taking smooth duals (which is exact (Lemma 2.10 of Bushnell-Henniart) and preserves irreducibles (by checking on $V^K$)) to get 
$$\xymatrix{0 \ar[r] & \mathrm{St}_G^\vee \ar[r] & \Ind_B^G \delta_B^{-1} \ar[r] & 1_G \ar[r] & 0}$$
The Proposition applied to $\chi = 1$ then implies
$$\mathrm{St}_G \cong \mathrm{St}_G^\vee.$$

\begin{notn}
    Define normalised induction by
    $$\iota_B^G \sigma = \Ind_B^G (\delta_B^{-1/2} \otimes \sigma).$$
    This has the benefit that $(\iota_B^G \sigma)^\vee \cong \iota_B^G \sigma^\vee$.
\end{notn}

\begin{thm}[Classification Theorem]
    The following are all the isomorphism classes of principal series representations of $G$:
    \begin{itemize}
        \item the irreducible induced representations $\iota_B^G \chi$ when $\chi \neq \phi \delta_B^{\pm 1/2}$ for a character $\phi$ of $F^\times$.
        \item the one-dimensional representations $\phi \circ \det$ for $\phi$ a character of $F^\times$.
        \item the twists of Steinberg (special representations) $\phi \mathrm{St}_G$ for $\phi$ a character of $F^\times$.
    \end{itemize}
    These are all distinct isomorphism classes except in the first case where $\iota_B^G \chi \cong \iota_B^G \chi^w$.
\end{thm}